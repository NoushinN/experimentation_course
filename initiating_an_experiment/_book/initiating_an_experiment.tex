\PassOptionsToPackage{unicode=true}{hyperref} % options for packages loaded elsewhere
\PassOptionsToPackage{hyphens}{url}
%
\documentclass[]{book}
\usepackage{lmodern}
\usepackage{amssymb,amsmath}
\usepackage{ifxetex,ifluatex}
\usepackage{fixltx2e} % provides \textsubscript
\ifnum 0\ifxetex 1\fi\ifluatex 1\fi=0 % if pdftex
  \usepackage[T1]{fontenc}
  \usepackage[utf8]{inputenc}
  \usepackage{textcomp} % provides euro and other symbols
\else % if luatex or xelatex
  \usepackage{unicode-math}
  \defaultfontfeatures{Ligatures=TeX,Scale=MatchLowercase}
\fi
% use upquote if available, for straight quotes in verbatim environments
\IfFileExists{upquote.sty}{\usepackage{upquote}}{}
% use microtype if available
\IfFileExists{microtype.sty}{%
\usepackage[]{microtype}
\UseMicrotypeSet[protrusion]{basicmath} % disable protrusion for tt fonts
}{}
\IfFileExists{parskip.sty}{%
\usepackage{parskip}
}{% else
\setlength{\parindent}{0pt}
\setlength{\parskip}{6pt plus 2pt minus 1pt}
}
\usepackage{hyperref}
\hypersetup{
            pdftitle={Initiating an Experiment in the public service},
            pdfauthor={Noushin Nabavi},
            pdfborder={0 0 0},
            breaklinks=true}
\urlstyle{same}  % don't use monospace font for urls
\usepackage{longtable,booktabs}
% Fix footnotes in tables (requires footnote package)
\IfFileExists{footnote.sty}{\usepackage{footnote}\makesavenoteenv{longtable}}{}
\usepackage{graphicx,grffile}
\makeatletter
\def\maxwidth{\ifdim\Gin@nat@width>\linewidth\linewidth\else\Gin@nat@width\fi}
\def\maxheight{\ifdim\Gin@nat@height>\textheight\textheight\else\Gin@nat@height\fi}
\makeatother
% Scale images if necessary, so that they will not overflow the page
% margins by default, and it is still possible to overwrite the defaults
% using explicit options in \includegraphics[width, height, ...]{}
\setkeys{Gin}{width=\maxwidth,height=\maxheight,keepaspectratio}
\setlength{\emergencystretch}{3em}  % prevent overfull lines
\providecommand{\tightlist}{%
  \setlength{\itemsep}{0pt}\setlength{\parskip}{0pt}}
\setcounter{secnumdepth}{5}
% Redefines (sub)paragraphs to behave more like sections
\ifx\paragraph\undefined\else
\let\oldparagraph\paragraph
\renewcommand{\paragraph}[1]{\oldparagraph{#1}\mbox{}}
\fi
\ifx\subparagraph\undefined\else
\let\oldsubparagraph\subparagraph
\renewcommand{\subparagraph}[1]{\oldsubparagraph{#1}\mbox{}}
\fi

% set default figure placement to htbp
\makeatletter
\def\fps@figure{htbp}
\makeatother

\usepackage{booktabs}
\usepackage[]{natbib}
\bibliographystyle{plainnat}

\title{Initiating an Experiment in the public service}
\author{Noushin Nabavi}
\date{2020-09-01}

\begin{document}
\maketitle

{
\setcounter{tocdepth}{1}
\tableofcontents
}
\hypertarget{preface}{%
\chapter{Preface}\label{preface}}

This is a repository to house course materials related to module 2 of Government of Canada's Experimentation Course, \texttt{Initiating\ an\ Experiment}.

\hypertarget{prerequisites}{%
\section{Prerequisites}\label{prerequisites}}

An interest in experimentation and apetite for informed consumation of evidence. Additionally, being part of a team of knowledgeable partners with interests in experimentation with policies, programs, and services, helps with developing contexts.

\hypertarget{learning-objectives}{%
\section{Learning objectives}\label{learning-objectives}}

\begin{itemize}
\tightlist
\item
  Identify the steps needed before starting an experimental project
\item
  Explain the steps involved when deciding to experiment
\item
  Define the problems before running an experiment
\item
  Design a research experiment
\item
  Provide examples of Experimentation
\end{itemize}

\hypertarget{outline}{%
\section{Outline}\label{outline}}

\begin{longtable}[]{@{}ll@{}}
\toprule
Chapter & Title\tabularnewline
\midrule
\endhead
1 & Before initiating an experiment\tabularnewline
2 & Components of an experimental project\tabularnewline
3 & Mechanics of endorsement\tabularnewline
4 & Case study\tabularnewline
5 & References\tabularnewline
\bottomrule
\end{longtable}

\hypertarget{experiment-components}{%
\chapter{Experiment components}\label{experiment-components}}

\hypertarget{endorsements}{%
\section{Endorsements}\label{endorsements}}

\begin{itemize}
\tightlist
\item
  executive support
\end{itemize}

\hypertarget{collaboration-agreements}{%
\section{Collaboration agreements}\label{collaboration-agreements}}

\begin{itemize}
\tightlist
\item
  team support
\end{itemize}

\hypertarget{funding-models-and-budget-targets}{%
\section{Funding models and budget targets}\label{funding-models-and-budget-targets}}

\hypertarget{past-research-and-literature-search}{%
\section{Past research and literature search}\label{past-research-and-literature-search}}

\hypertarget{defining-relevant-and-specific-questions}{%
\section{Defining relevant and specific questions}\label{defining-relevant-and-specific-questions}}

\begin{itemize}
\tightlist
\item
  context and significance
\end{itemize}

\hypertarget{project-scoping-and-problem-statement}{%
\section{Project scoping and problem statement}\label{project-scoping-and-problem-statement}}

\begin{itemize}
\tightlist
\item
  objectives and aims
\end{itemize}

\hypertarget{experimental-design}{%
\section{Experimental design}\label{experimental-design}}

\begin{itemize}
\tightlist
\item
  controlled experimentation
\end{itemize}

\hypertarget{project-timelines-and-milestones}{%
\section{Project timelines and milestones}\label{project-timelines-and-milestones}}

\hypertarget{communication-matrix}{%
\section{Communication matrix}\label{communication-matrix}}

\begin{itemize}
\tightlist
\item
  stakeholder consensus
\end{itemize}

\hypertarget{decision-to-experiment}{%
\chapter{Decision to experiment}\label{decision-to-experiment}}

\hypertarget{deciding-to-experiment}{%
\section{Deciding to experiment}\label{deciding-to-experiment}}

\begin{itemize}
\tightlist
\item
  Do you need to experiment? Why or why not?
\item
  Find a behaviour or object to test and think short term
\item
  Keep it simple
\item
  Start with a proof-of-concept test
\item
  Have control and treatment groups (e.g.~randomization)
\item
  When results are in slice the data
\item
  Try out of the box thinking
\item
  Measure everything that matters
\item
  Look for natural experiments
\end{itemize}

\hypertarget{steps-involved-in-designing-an-experiment}{%
\section{Steps involved in designing an experiment}\label{steps-involved-in-designing-an-experiment}}

\hypertarget{step-1-choose-a-topic}{%
\subsection{Step 1: Choose a topic}\label{step-1-choose-a-topic}}

\begin{itemize}
\tightlist
\item
  Ask yourself the following question: 1. What do I find interesting about the subject? 2. What is known about the subject? 3. What is missing and the gaps?
\end{itemize}

\hypertarget{step-2-narrow-the-topic}{%
\subsection{Step 2: Narrow the topic}\label{step-2-narrow-the-topic}}

\begin{itemize}
\tightlist
\item
  Ask yourself the following questions: 1. What do you need to know more about on the topic? 2. Are you interested in social, political, economic, gender, religious issues related to your topic? (General example) - Find a ``slant'' on your topic;
\item
  Will the results reveal something new or unexpected?
\item
  What is in scope and what is out of scope?
\item
  Clearly define hypotheses and explicitly state research questions
\end{itemize}

\hypertarget{step-3-find-resources}{%
\subsection{Step 3: Find Resources}\label{step-3-find-resources}}

\begin{itemize}
\tightlist
\item
  Use the keywords you have compiled and use them to search for books in Library Catalogs or articles in online article databases.
\item
  Team expertise
\end{itemize}

\hypertarget{step-4-solicit-feedback-and-collaboration}{%
\subsection{Step 4: Solicit feedback and collaboration}\label{step-4-solicit-feedback-and-collaboration}}

\begin{itemize}
\tightlist
\item
  Make sure the question is one that other people can get behind and support
\item
  Establish collaboration agreements and executive buy-in
\item
  Peer-review for clarity, scientific accuracy, and feasibility
\item
  Does the team have the expertise required to complete the project? If not, who else needs to be on team
\end{itemize}

\hypertarget{experimental-design-cycle}{%
\section{Experimental design cycle}\label{experimental-design-cycle}}

\begin{itemize}
\tightlist
\item
  Problem statement
\item
  Question
\item
  Research
\item
  Hypotheses (It\ldots{}.then\ldots{})
\item
  Identify controls and experimental group as well as interventions/treatments
\item
  A control group is a group of `test' items in an experiment. The control group will be used to compare with the experimental group
\item
  The control group doesn't get the treatment
\item
  An experimental group is the group(s) of test items where only one change (called the experimental or independent variable) has been made
\item
  The experimental group gets the treatment
\item
  The experimental group may have dependent or independent variables
\item
  Sample size
\item
  Maximize sample size: the larger the number of test items the more accurate the estimate
\item
  Use representative groups: the samples must reflect the natural variation in the population. Use random or systematic sampling to reduce inherent bias in data.
\item
  Determine outcome measures and visualization of outcome
\item
  Independent variable or the factor that is manipulated or changed in placed on x-axis when grouping
\item
  Dependent variable or the factor that is being measured is placed on y-axis during grouping
\item
  Identify sources of error
\item
  Mind the constants: the conditions that are kept the same for control and experimental groups
\item
  Not controlling for factors or parameters that are kept the same in both control and experimental groups can result in error
\item
  Report back and adapt
\item
  Data analysis
\item
  Implement intervention and measure outcomes and impacts
\end{itemize}

\hypertarget{mechanics-of-endoresement}{%
\chapter{Mechanics of endoresement}\label{mechanics-of-endoresement}}

\hypertarget{requirements}{%
\section{requirements}\label{requirements}}

\begin{itemize}
\tightlist
\item
  Use communication matrix to relay information to stakeholders/executive
\item
  Importance: Why do we need to answer this scientific question now?
\item
  Novelty: Has this question been answered? Has it been attempted?
\item
  Impact: What's the risk, and what's the potential upside?
\item
  Design: Is the design of the experiment sound? how?
\item
  Qualifications: What makes this researcher/research uniquely qualified?
\end{itemize}

\hypertarget{communication-matrix}{%
\section{Communication matrix}\label{communication-matrix}}

\begin{itemize}
\tightlist
\item
  Establish a communication matrix
\item
  Experiment status project updates to whom and how
\item
  Foster a culture of experimentation in the organization
  Explain the business value of experimentation to decision-makers
\item
  Establish accountabilities
\item
  Frequency and content of communications
\item
  Structural report updates
\end{itemize}

\hypertarget{code-of-conduct}{%
\section{Code of conduct}\label{code-of-conduct}}

\begin{itemize}
\tightlist
\item
  Ethical considerations
\item
  Information sharing agreements and memorandums of understanding
\item
  Identify situations where experimentations are appropriate and relevant
\item
  Check Institutional Review Board (IRB) approval requirements prior to the launch of the experiment
\item
  Perform privacy impact assessments
\item
  Get ethics committee approvals
\end{itemize}

\hypertarget{case-study}{%
\chapter{Case study}\label{case-study}}

\begin{itemize}
\item
  City of Vancouver Solution Lab's \href{https://vancouver.ca/files/cov/navigating-complexity-solutions-lab.pdf}{Principles of Experimentation} (pg. 38), adapted from Nesta's \href{https://www.nesta.org.uk/toolkit/playbook-for-innovation-learning/}{Innovation Playbook}, 2018
\item
  Nesta's Competency \href{https://media.nesta.org.uk/documents/Nesta_CompetencyFramework_Guide_July2019.pdf}{Framework} for Experimental Problem Solving (pg. 2)
\item
  States of Change's \href{https://states-of-change.org/assets/images/StatesofChange_Curriculum_Craft.png}{Core Elements of Innovation} Craft
\item
  Tatyana Mamut's eight \href{https://gww.gov.bc.ca/sites/default/files/group/file/2019/0207/remixtatyanamamutleadingacultureofinnovationlowres.pdf}{Innovation Elements} (pg. 6)
\item
  The Moment's \href{https://info.themoment.is/innovationculture}{Culture Scan}
\item
  Innovation Designer \href{https://cdn2.hubspot.net/hubfs/3903042/themoment_InnovationDesignersCapabilityMap.pdf}{Capability Map}
\end{itemize}

\end{document}
